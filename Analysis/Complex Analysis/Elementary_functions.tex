\documentclass{article}
\usepackage{amsthm}
\usepackage{amssymb}
\usepackage{amsmath}
\title{Elementary Complex Functions}
\author{Yuxin Gong}
\date{April 2025}
\linespread{1.6}
\begin{document}
\maketitle
\section{Introduction}
In this note, we will introduce some basic functions $f : \mathbb{C} \to \mathbb{C}$.
Things seem to become complicated when they are written complex numbers.
However, it turns out many things in complex will become much simpler in 
real functions. For example, if we say some functions is holomorphic, it 
will then be infinitly differentiable. Anyway, those are strayed off topic we gonna
discuss today. Before discussing the topic, let me quote a sentence that might
be instructive for complex analysis.
\begin{quote}
    \textit{Sometimes the quickest way forward is the long road around.}
\end{quote}
\section{Problems}
Consider the following function
$$
f : \mathbb{C} \to \mathbb{C}\, , \, f(z) = z^\alpha
$$
Think about the cases where $z$ is a real number. If $\alpha$ is an integer, we have 
nothing to worry about cuz it is just some self-multiplication. In real analysis, we 
know that $x^{\alpha}$ is continuous for every real power on the region $(0, \infty)$.
However, in complex analysis, we should really think about a general way to 
define such function.
\section{Exponential Functions}
In general, there are two ways to define $e^z$, one defines using the series
$$
e^z = \sum_{k = 0}^{\infty} \frac{z^k}{k !}
$$
Alternatively, one can define $e^z = e^x \cos(y) + e^x \sin(y)i$. They are equivalenet, 
having benefits for different purposes. Let's use the second one here.
Using Cauchy-Riemann equations, 
\begin{align}
\frac{\partial }{\partial x}e^x \cos(y) &= \frac{\partial}{\partial y} e^x \sin(y) \\
\frac{\partial }{\partial y}e^x \cos(y) &= -\frac{\partial}{\partial x} e^x \sin(y)
\end{align}
$e^z$ is holomorphic for all $z \in \mathbb{C}$. It has some property we expect
$$
e^{z_1 + z_2} = e^{z_1} \cdot e^{z_2} 
$$
\end{document}